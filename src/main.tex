%! TEX root = main.tex
% Stile del documento
\documentclass[12pt,openany]{amsbook}
\usepackage[T1]{fontenc}
\usepackage[utf8]{inputenc}
\usepackage[a4paper]{geometry}
\geometry{verbose,tmargin=1.8cm,bmargin=2cm,lmargin=2cm,rmargin=2cm,headheight=1.8cm,headsep=1cm,footskip=1cm}
% Localizzazione (traduce nomi automatici nella lingua selezionata)
\usepackage[italian]{babel}
% Altri pacchetti per la formattazione
\usepackage{lmodern}
\usepackage{hyphenat}
\usepackage{amsmath}
\usepackage{amsthm}
\usepackage{amssymb}
\usepackage{mathrsfs}
\usepackage{mathtools}
\usepackage{cancel}
\usepackage{enumerate}
\usepackage{graphicx}
 % Impostazioni dei metadati del PDF
\usepackage[unicode=true, bookmarks=true, bookmarksnumbered=true, bookmarksopen=true, bookmarksopenlevel=1, breaklinks=false, pdfborder={0 0 0}, pdfborderstyle={}, backref=page, colorlinks=false]
{hyperref}
\hypersetup{
    pdftitle={Eserciziario di Sicurezza delle Reti},
    pdfauthor={Lorenzo Prosseda},
    pdfsubject={Corso di Sicurezza delle reti del prof. Bregni, Politecnico di Milano 2019\---2020},
    pdfkeywords={sicurezza, reti, crittografia}
}
% Percorso della cartella contenente le immagini
\graphicspath{ {../img/} }
% Impostazioni di stile per gli ambienti teorema, definizione, esempio, ecc
\theoremstyle{plain}
\newtheorem{thm}{Teorema}[section] % resetta numerazione ad ogni capitolo
\renewcommand{\thethm}{\arabic{chapter}.\arabic{section}.\arabic{thm}} % numerazione ambienti teorema
\numberwithin{equation}{section} % numerazione equazioni
% Definizione ambienti personalizzati
\theoremstyle{definition}
\newtheorem{defn}[thm]{Definizione}
\newtheorem{exmp}[thm]{Esempio}
\newtheorem{obsv}[thm]{Osservazione}
\theoremstyle{remark}
\newtheorem{note}[thm]{Nota}
% Numerazione personalizzata sezioni e sottosezioni
\renewcommand\thesection{\thechapter.\arabic{section}}
\renewcommand\thesubsection{\thesection.\arabic{subsection}}
% Comando trattino "non spezzante" evita sillabazione sul trattino
\newcommand\nbdash{\nobreakdash-\hspace{0pt}}
% Non spezzare e seguenti parole:
\hyphenation{Definizione}
\hyphenation{Osservazione}
\hyphenation{Equazione}
\hyphenation{Teorema}
\hyphenation{Nota}
% Solleva posizione lettera greca chi minuscola
\usepackage{letltxmacro}
\LetLtxMacro{\lowchi}{\chi}
\renewcommand\chi{\text{\raisebox{2pt}{$\lowchi$}}}
% Usa spaziatura corretta attorno a modulo
\renewcommand\mod{\,\mathrm{mod}\,}

% Comando per congruenza in modulo
\newcommand\cgm[3]{%
  #1\equiv\,#2\,(\mathrm{mod}\,#3)%
}

% Comando per ambiente problema e soluzione
\newcounter{problem}
\newcounter{solution}
\newcommand\Problem[1]{%
  \stepcounter{problem}%
  \bigskip
  \textbf{Esercizio \theproblem.}~\emph{(#1)}~%
  \setcounter{solution}{0}%
}
\newcommand\TheSolution{%
  \textbf{Soluzione}\\%
}
\newcommand\ASolution[1]{%
  \stepcounter{solution}%
  \bigskip
  \textbf{Soluzione \thesolution:}\hfill\emph{(#1)}\\%
}

% In questo ambiente risiede il contenuto del documento
\begin{document}
    % Pagina del titolo
    \input{title}

    % Genera indice di capitoli e sezioni
    \tableofcontents

    % Includi file dei capitoli
    \part{Esercizi}
    %! TEX root = main.tex

\chapter{Aritmetica modulare}

    \bigskip
    \section{Potenze}
        \Problem{11/07/2013}
        Calcolare $8751423^{65536} \mod 131072$

        \Problem{04/07/2014}
        Siano noti $n=p\cdot q =314869$ prodotto di due primi distinti $p$ e $q$, e $\varphi(n)=313740$: calcolare 
        $p$ e $q$.

    \bigskip
    \section{Radici}
        \Problem{27/02/2013}
        L'equazione $\cgm{x^2}{5}{239}$ ha soluzione? Se la risposta è affermativa, calcolarne 
        le radici, altrimenti risolvere $\cgm{x^2}{-5}{239}$.

        \Problem{18/07/2014}
        L'equazione $\cgm{x^2}{25}{223}$ ha soluzione? Se la risposta è affermativa, calcolarne 
        le radici, altrimenti risolvere $\cgm{x^2}{-25}{223}$.

        \Problem{31/08/2019}
        L'equazione $\cgm{x^2}{11}{359}$ ha soluzione? Se la risposta è affermativa, calcolarne 
        le radici, altrimenti risolvere $\cgm{x^2}{-11}{359}$.

    \bigskip
    \section{Inverso}
        \Problem{08/02/2013}
        Calcolare l'inverso $a^{-1}$ di $\cgm{a}{10}{999}$ applicando il \emph{teorema di Eulero} 
        e quindi l'algoritmo \emph{square \& multiply} per il calcolo della potenza.

        \Problem{11/07/2013}
        Calcolare $500^{-1} \mod 977$ per mezzo dell'\emph{algoritmo di Euclide esteso}.

        \Problem{04/07/2014}
        Applicando il \emph{teorema cinese del resto}, trovare la congruenza equivalente a \[
        \begin{cases}
            \cgm{x}{6}{15} \\
            \cgm{x}{15}{16}
        \end{cases}
        ,\] verificando le ipotesi di questo teorema.

        \Problem{17/09/2014}
        Calcolare $23^{-1} \mod 317$ per mezzo dell'\emph{algoritmo di Euclide esteso}.

        \Problem{03/07/2015}
        Calcolare $3\cdot 55^{-1} \mod 317$ per mezzo dell'\emph{algoritmo di Euclide esteso}.

    \bigskip
    \section{Residui quadratici}
        \Problem{08/02/2013}
        Trovare ed elencare i residui quadratici $a_q$ dell'insieme $\mathbb{Z}^{*}_p$ per $p=17$, ovvero gli 
        elementi tali che $\forall a\in \mathbb{Z}^{*}_p \,:\, \cgm{a}{(\pm b)^2}{p} \land b\in \mathbb{Z}^{*}_p$.

        \Problem{27/02/2013}
        \begin{enumerate}
            \item Quali sono gli elementi primitivi $\alpha$ dell'insieme $\mathbb{Z}^{*}_{17}$?
            \item Quanti sono gli elementi primitivi di $\mathbb{Z}^{*}_{17}$?
            \item Trovare ed elencare in ordine gli elementi primitivi di $\mathbb{Z}^{*}_{17}$.
            \item Qual è l'ordine dell'elemento $\alpha =11$?
            \item Qual è l'ordine dell'elemento $\alpha =13$?
        \end{enumerate}

        \Problem{14/02/2014}
        \begin{enumerate}
            \item Trovare ed elencare in ordine crescente gli elementi primitivi di $\mathbb{Z}^{*}_{19}$
            \item Qual è l'ordine dell'elemento $\alpha =11$?
            \item Qual è l'ordine dell'elemento $\alpha =13$?
            \item Trovare ed elencare in ordine crescente i residui quadratici di $\mathbb{Z}^{*}_{19}$, ovvero gli 
                elementi tali che $\forall a\in \mathbb{Z}^{*}_{19} \,:\, \cgm{a}{(\pm b)^2}{19} \land 
                b\in \mathbb{Z}^{*}_{19}$.
        \end{enumerate}

        \Problem{17/09/2014}
        \begin{enumerate}
            \item Cos'è un elemento primitivo $\alpha \in \mathbb{Z}^{*}_p$?
            \item Qual è l'ordine dell'elemento $\alpha =11$? È un elemento primitivo di $\mathbb{Z}^{*}_{23}$?
            \item Qual è l'ordine dell'elemento $\alpha =13$? È un elemento primitivo di $\mathbb{Z}^{*}_{23}$?
            \item Trovare ed elencare in ordine crescente i residui quadratici di $\mathbb{Z}^{*}_{23}$, ovvero gli 
                elementi tali che $\forall a\in \mathbb{Z}^{*}_{23} \,:\, \cgm{a}{(\pm b)^2}{23} \land 
                b\in \mathbb{Z}^{*}_{23}$.
        \end{enumerate}

        \Problem{02/03/2015}
        \begin{enumerate}
            \item Cos'è un residuo quadratico dell'insieme $\mathbb{Z}^{*}_p$?
            \item Dopo aver calcolato tutti i residui quadratici di $\mathbb{Z}^{*}_{19}$, dire quali sono le radici 
                quadrate di  $-2 \mod 19$.
        \end{enumerate}

        \Problem{03/07/2015}
        \begin{enumerate}
            \item Cos'è un elemento primitivo $\alpha \in \mathbb{Z}^{*}_p$?
            \item Trovare ed elencare in ordine crescente i primi quattro elementi primitivi di $\mathbb{Z}^{*}_{31}$.
            \item Qual è l'ordine dell'elemento $\alpha =6$?
        \end{enumerate}

        \Problem{09/02/2017}
        \begin{enumerate}
            \item Cos'è un elemento primitivo $\alpha \in \mathbb{Z}^{*}_p$?
            \item Quanti sono gli elementi primitivi di $\mathbb{Z}^{*}_{61}$?
            \item Qual è l'ordine dell'elemento $\alpha =3$?
        \end{enumerate}

    \bigskip
    \section{Logaritmo discreto}
        \Problem{08/02/2013}
        Calcolare il logaritmo discreto, soluzione dell'equazione $\cgm{\alpha^x}{\beta}{p}$ per 
        $p=23,\, \alpha =5,\, \beta =21$, applicando l'algoritmo \emph{baby step, giant step}; verificare per 
        prima cosa se esiste con certezza una soluzione.

        \Problem{03/03/2014}
        Trovare le soluzioni dell'equazione $\cgm{13^x}{2}{19}$.

        \Problem{17/07/2015}
        Calcolare il logaritmo discreto, soluzione dell'equazione $\cgm{\alpha^x}{\beta}{p}$ per 
        $p=89,\, \alpha =7,\, \beta =53$, applicando l'algoritmo \emph{baby step, giant step}; verificare per 
        prima cosa se esiste con certezza una soluzione.

        \Problem{23/09/2016}
        Calcolare il logaritmo discreto, soluzione dell'equazione $\cgm{\alpha^x}{\beta}{p}$ per 
        $p=103,\, \alpha =12,\, \beta =20$, applicando l'algoritmo \emph{baby step, giant step}; verificare per prima 
        cosa se esiste con certezza una soluzione. Per quanti valori di $b\in \mathbb{Z}^{*}_p$ l'equazione 
        ammette soluzione?

        \Problem{06/07/2018}
        Calcolare il logaritmo discreto, soluzione dell'equazione $\cgm{\alpha^x}{\beta}{p}$ per 
        $p=139,\, \alpha =3,\, \beta =40$, applicando l'algoritmo \emph{baby step, giant step}; verificare per prima 
        cosa se esiste con certezza una soluzione. Per quanti valori di $b\in \mathbb{Z}^{*}_p$ l'equazione 
        ammette soluzione?

    \bigskip
    \section{Fattorizzazione}
        \Problem{04/07/2016}
        Trovare i fattori primi di $n=20687$ attraverso l'\emph{algoritmo di fattorizzazione $p-1$ di Pollard} con 
        base $a=2$.

        \Problem{21/07/2016}
        Trovare i fattori primi di $n=1083617$ attraverso il \emph{metodo di fattorizzazione di Fermat}.

        \Problem{09/02/2017}
        Trovare i fattori primi di $n=14803$ attraverso l'\emph{algoritmo di fattorizzazione $p-1$ di Pollard} con 
        base $a=2$.

        \Problem{23/09/2016}
        Trovare i fattori primi di $n=2047697$ attraverso il \emph{metodo di fattorizzazione di Fermat}.

        \Problem{30/08/2018}
        Trovare i fattori primi di $n=38191$ attraverso l'\emph{algoritmo di fattorizzazione $p-1$ di Pollard} con 
        base $a=2$.

        \Problem{11/02/2019}
        Trovare i fattori primi di $n=13843$ attraverso l'\emph{algoritmo di fattorizzazione $p-1$ di Pollard} con 
        base $a=2$.

        \Problem{26/07/2019}
        Trovare i fattori primi di $n=1734389$ attraverso il \emph{metodo di fattorizzazione di Fermat}.

        \Problem{31/08/2019}
        Trovare i fattori primi di $n=19549$ attraverso l'\emph{algoritmo di fattorizzazione $p-1$ di Pollard} con 
        base $a=2$.

    %! TEX root = main.tex
% Capitolo 2

\chapter{Crittografia a chiave asimmetrica}

    \bigskip
    \section{Crittosistema a chiave pubblica RSA}
        \Problem{11/07/2013}
        Bob adotta il sistema di cifratura a chiave pubblica RSA, pubblicando il modulo $n=323$ e 
        l'esponente di cifratura $e=17$.
        \begin{enumerate}
            \item Verificare la correttezza dei dati forniti sulla base delle ipotesi del metodo RSA.
            \item Alice trasmette il messaggio cifrato $C=55$; calcolare il messaggio in chiaro $P$.
            \item Alice vuole inviare il messaggio in chiaro $P=55$; calcolare il corrispondente messaggio 
                cifrato $C$.
        \end{enumerate}

        \Problem{04/07/2014}
        Bob adotta il sistema di cifratura a chiave pubblica RSA, pubblicando il modulo $n=2867$ e 
        l'esponente di cifratura $e=677$.
        \begin{enumerate}
            \item Verificare la correttezza dei dati forniti sulla base delle ipotesi del metodo RSA.
            \item Alice trasmette il messaggio cifrato $C=16$; calcolare il messaggio in chiaro $P$.
        \end{enumerate}

        \Problem{17/09/2014}
        Bob adotta il sistema di cifratura a chiave pubblica RSA, pubblicando il modulo $n=3953$ e 
        l'esponente di cifratura $e=1225$.
        \begin{enumerate}
            \item Verificare la correttezza dei dati forniti sulla base delle ipotesi del metodo RSA.
            \item Alice trasmette il messaggio cifrato $C=100$; calcolare il messaggio in chiaro $P$.
        \end{enumerate}

        \Problem{02/03/2015}
        Bob adotta il sistema di cifratura a chiave pubblica RSA, pubblicando il modulo $n=8791$ e 
        l'esponente di cifratura $e=6243$.
        \begin{enumerate}
            \item Verificare la correttezza dei dati forniti sulla base delle ipotesi del metodo RSA.
            \item Alice trasmette il messaggio cifrato $C=10$; calcolare il messaggio in chiaro $P$.
        \end{enumerate}

        \Problem{09/09/2016}
        Bob adotta il sistema di cifratura a chiave pubblica RSA, pubblicando il modulo $n=4189$ e 
        l'esponente di cifratura $e=3707$.
        \begin{enumerate}
            \item Verificare la correttezza dei dati forniti sulla base delle ipotesi del metodo RSA.
            \item Alice trasmette il messaggio cifrato $C=50$; calcolare il messaggio in chiaro $P$.
        \end{enumerate}

        \Problem{09/09/2016}
        Bob adotta il sistema di cifratura a chiave pubblica RSA, pubblicando il modulo $n=667$ e 
        l'esponente di cifratura $e=5$; assumiamo che questi dati rispettino le ipotesi del metodo RSA.
        \begin{enumerate}
            \item Alice trasmette il messaggio cifrato $C=523$; calcolare il messaggio in chiaro $P$ tramite 
                l'attacco a RSA con testo in chiaro corto, supponendo che $P$ sia il prodotto di due interi 
                $i<10, k<10 \,:\, P=i\cdot k$. Non si deve tentare di fattorizzare $n$ e calcolare $\varphi(n)$ 
                per decifrare $P$.
        \end{enumerate}

        \Problem{06/07/2018}
        Bob adotta il sistema di cifratura a chiave pubblica RSA, pubblicando il modulo $n=5251$ e 
        l'esponente di cifratura $e=1021$; assumiamo che questi dati rispettino le ipotesi del metodo RSA.
        \begin{enumerate}
            \item Verificare la correttezza dei dati forniti sulla base delle ipotesi del metodo RSA.
            \item Oscar vuole firmare messaggi impersonando Bob sulla base dei dati pubblici: cosa deve 
                fare? Calcolare la firma di Bob $A(m)$ per il messaggio $m=112$.
        \end{enumerate}

        \Problem{11/02/2019}
        Bob adotta il sistema di cifratura a chiave pubblica RSA, pubblicando il modulo $n=5141$ e 
        due esponenti di cifratura $e_1=1521,\,e_2=1523$.
        \begin{enumerate}
            \item Verificare la correttezza dei dati forniti sulla base delle ipotesi del metodo RSA; 
                scegliere il valore corretto tra i due esponenti $e_1$ ed $e_2$.
            \item Oscar vuole firmare messaggi impersonando Bob sulla base dei dati pubblici: cosa deve 
                fare? Calcolare la firma di Bob $A(m)$ per il messaggio $m=7$ (usare il valore di $e$ 
                corretto, ottenuto nel quesito precedente).
        \end{enumerate}

        \Problem{26/02/2019}
        Bob adotta il sistema di cifratura a chiave pubblica RSA, pubblicando il modulo $n=9797$ e 
        due esponenti di cifratura $e_1=2087,\,e_2=2097$.
        \begin{enumerate}
            \item Verificare la correttezza dei dati forniti sulla base delle ipotesi del metodo RSA; 
                scegliere il valore corretto tra i due esponenti $e_1$ ed $e_2$.
            \item Oscar vuole firmare messaggi impersonando Bob sulla base dei dati pubblici: cosa deve 
                fare? Calcolare la firma di Bob $A(m)$ per il messaggio $m=111$ (usare il valore di $e$ 
                corretto, ottenuto nel quesito precedente).
        \end{enumerate}

    \bigskip
    \section{Pattuizione della chiave Diffie-Hellman}
        \Problem{25/09/2013}
        Alice e Bob adottano il protocollo Diffie-Hellman per lo scambio della chiave; Alice pubblica 
        $p=107,\, \alpha=4$ (da verificare) e sceglie $x=23$ segreto, mentre Bob sceglie $y=3$ segreto.
        \begin{enumerate}
            \item Verificare la correttezza dei dati forniti secondo le ipotesi del protocollo Diffie-Hellman 
                per lo scambio della chiave; se $\alpha=4$ non risultasse una scelta valida, Alice 
                pubblicherebbe invece $\alpha=5$ (da verificare); se anche questa scelta non fosse valida, Alice e Bob 
                rinuncerebbero a continuare (l'esercizio termina).
            \item Calcolare i messaggi scambiati e la chiave $K$ condivisa al termine.
        \end{enumerate}

        \Problem{03/03/2014}
        Alice e Bob adottano il protocollo Diffie-Hellman per lo scambio della chiave; Alice pubblica 
        $p=139,\, \alpha=4$ (da verificare) e sceglie $x=57$ segreto, mentre Bob sceglie $y=12$ segreto. Oscar 
        si intromette e sceglie $z=2$ segreto (attacco man in the middle).
        \begin{enumerate}
            \item Verificare la correttezza dei dati forniti secondo le ipotesi del protocollo Diffie-Hellman 
                per lo scambio della chiave; se $\alpha=4$ non risultasse una scelta valida, Alice 
                pubblicherebbe invece $\alpha \in \{3,\,5,\,6,\,7,\,8,\,9,\,10\}$ (da verificare); se anche 
                questa scelta non fosse valida, Alice e Bob rinuncerebbero a continuare (l'esercizio termina).
            \item Calcolare i messaggi scambiati tra Alice, Bob e Oscar e le chiavi $K_A,\,K_B$ condivise tra 
                Oscar e Alice e tra Oscar e Bob.
        \end{enumerate}

        \Problem{21/07/2016}
        Alice e Bob adottano il protocollo Diffie-Hellman per lo scambio della chiave; Alice pubblica 
        $p=47,\, \alpha=4$ (da verificare) e sceglie $x\in [1,\,p-2]$ segreto, mentre Bob sceglie $y\in [1,\,p-2]$ 
        segreto.
        \begin{enumerate}
            \item Verificare la correttezza dei dati forniti secondo le ipotesi del protocollo Diffie-Hellman 
                per lo scambio della chiave; se $\alpha=4$ non risultasse una scelta valida, Alice 
                pubblicherebbe invece $\alpha \in \{3,\,5,\,6,\,7,\,8,\,9\}$ (da verificare); se anche 
                questa scelta non fosse valida, Alice e Bob rinuncerebbero a continuare (l'esercizio termina).
            \item Oscar osserva i seguenti messaggi scambiati da Alice e Bob: \[
            \begin{array}{ll}
                \text{Alice} \rightarrow \text{Bob} & \alpha^{x} \equiv\,22\,(\mathrm{mod}\,p)\\
                \text{Alice} \leftarrow \text{Bob} & \alpha^{y} \equiv\,23\,(\mathrm{mod}\,p)
            \end{array}
            \] Sulla base di queste informazioni, calcolare gli esponenti $x$ e $y$ e la chiave $K_{AB}$.
        \end{enumerate}

        \Problem{09/02/2017}
        Alice e Bob adottano il protocollo Diffie-Hellman per lo scambio della chiave; Alice pubblica 
        $p=53,\, \alpha=4$ (da verificare) e sceglie $x\in [1,\,p-2]$ segreto, mentre Bob sceglie $y\in [1,\,p-2]$ 
        segreto.
        \begin{enumerate}
            \item Verificare la correttezza dei dati forniti secondo le ipotesi del protocollo Diffie-Hellman 
                per lo scambio della chiave; se $\alpha=4$ non risultasse una scelta valida, Alice 
                pubblicherebbe invece $\alpha \in \{5,\,6,\,7\}$ (da verificare); se anche questa scelta 
                non fosse valida, Alice e Bob rinuncerebbero a continuare (l'esercizio termina).
            \item Oscar osserva i seguenti messaggi scambiati da Alice e Bob: \[
            \begin{array}{ll}
                \text{Alice} \rightarrow \text{Bob} & \alpha^{x} \equiv\,26\,(\mathrm{mod}\,p)\\
                \text{Alice} \leftarrow \text{Bob} & \alpha^{y} \equiv\,42\,(\mathrm{mod}\,p)
            \end{array}
            \] Sulla base di queste informazioni, calcolare gli esponenti $x$ e $y$ e la chiave $K_{AB}$.
        \end{enumerate}

        \Problem{27/07/2018}
        Alice e Bob adottano il protocollo Diffie-Hellman per lo scambio della chiave; Alice pubblica 
        $p=199,\, \alpha=3$ (da verificare) e sceglie $x\in [1,\,p-2]$ segreto, mentre Bob sceglie $y\in [1,\,p-2]$ 
        segreto.
        \begin{enumerate}
            \item Verificare la correttezza dei dati forniti secondo le ipotesi del protocollo Diffie-Hellman 
                per lo scambio della chiave; se $\alpha=3$ non risultasse una scelta valida, Alice 
                pubblicherebbe invece $\alpha \in \{4,\,5\}$ (da verificare); se anche questa scelta 
                non fosse valida, Alice e Bob rinuncerebbero a continuare (l'esercizio termina).
            \item Oscar osserva i seguenti messaggi scambiati da Alice e Bob: \[
            \begin{array}{ll}
                \text{Alice} \rightarrow \text{Bob} & \alpha^{x} \equiv\,106\,(\mathrm{mod}\,p)\\
                \text{Alice} \leftarrow \text{Bob} & \alpha^{y} \equiv\,14\,(\mathrm{mod}\,p)
            \end{array}
            \] Sulla base di queste informazioni, calcolare gli esponenti $x$ e $y$ e la chiave $K_{AB}$.
        \end{enumerate}

        \Problem{02/07/2019}
        Alice e Bob adottano il protocollo Diffie-Hellman per lo scambio della chiave; Alice pubblica 
        $p=263,\, \alpha=13$ (da verificare) e sceglie $x\in [1,\,p-2]$ segreto, mentre Bob sceglie $y\in [1,\,p-2]$ 
        segreto.
        \begin{enumerate}
            \item Verificare la correttezza dei dati forniti secondo le ipotesi del protocollo Diffie-Hellman 
                per lo scambio della chiave; se $\alpha=13$ non risultasse una scelta valida, Alice 
                pubblicherebbe invece $\alpha=14$ (da verificare); se anche questa scelta 
                non fosse valida, Alice e Bob rinuncerebbero a continuare (l'esercizio termina).
            \item Oscar osserva i seguenti messaggi scambiati da Alice e Bob: \[
            \begin{array}{ll}
                \text{Alice} \rightarrow \text{Bob} & \alpha^{x} \equiv\,139\,(\mathrm{mod}\,p)\\
                \text{Alice} \leftarrow \text{Bob} & \alpha^{y} \equiv\,71\,(\mathrm{mod}\,p)
            \end{array}
            \] Sulla base di queste informazioni, calcolare gli esponenti $x$ e $y$ e la chiave $K_{AB}$.
        \end{enumerate}

    \bigskip
    \section{Crittosistema a chiave pubblica di El Gamal}
        \Problem{27/02/2013}
        Bob adotta il sistema di cifratura a chiave pubblica di El Gamal e pubblica i seguenti dati: \[
            p=127,\, \alpha=3,\, \beta= \alpha^{a} \,\mathrm{mod}\, p
        ;\] inoltre mantiene segreto l'esponente $a=98$.
        \begin{enumerate}
            \item Verificare la correttezza dei dati forniti, in base alle ipotesi del metodo di 
                El Gamal; se $\alpha=3$ non fosse una scelta valida, Bob userà $\alpha=6$ (da verificare). 
                Se nemmeno questa scelta risultasse valida, Bob rinuncia a proseguire (l'esercizio 
                termina); calcolare $\beta$.
            \item Alice estrae il numero casuale segreto (nonce) $k=110$ e spedisce il messaggio $P=33$; 
                calcolare il messaggio cifrato $C=(r,\,t)$.
            \item Bob, per un errore di trasmissione, riceve $C^{\prime}=(r^{\prime},\,t^{\prime})=(37,\,102)$; 
                calcolare il messaggio $P^{\prime}$ decifrato da Bob.
        \end{enumerate}

        \Problem{14/02/2014}
        Bob adotta il sistema di cifratura a chiave pubblica di El Gamal e pubblica i seguenti dati: \[
            p=127,\, \alpha=5,\, \beta= \alpha^{a} \,\mathrm{mod}\, p
        ;\] inoltre mantiene segreto l'esponente $a=64$.
        \begin{enumerate}
            \item Verificare la correttezza dei dati forniti, in base alle ipotesi del metodo di 
                El Gamal; se $\alpha=5$ non fosse una scelta valida, Bob userà $\alpha=6$ (da verificare). 
                Se nemmeno questa scelta risultasse valida, Bob rinuncia a proseguire (l'esercizio 
                termina); calcolare $\beta$.
            \item Alice estrae il numero casuale segreto (nonce) $k=110$ e spedisce il messaggio $P_1=12$; 
                calcolare il messaggio cifrato $C_1=(r_1,\,t_1)$.
            \item Alice estrae un nuovo nonce $k$, e usando questo stesso valore spedisce i messaggi 
                $P_2,\,P_3,\,P_4$; Oscar intercetta i seguenti messaggi cifrati, e scopre uno dei messaggi 
                in chiaro: \[
                \begin{array}{ll}
                    C_2=(r_2,\,t_2)=(96,\,13) & P_2=25\\
                    C_3=(r_3,\,t_3)=(96,\,121) & P_3=?\\
                    C_4=(r_4,\,t_4)=(96,\,62) & P_4=?
                \end{array}
                \] Calcolare $P_3$ e $P_4$.
        \end{enumerate}

        \Problem{03/03/2014}
        Bob adotta il sistema di cifratura a chiave pubblica di El Gamal e pubblica i seguenti dati: \[
            p=149,\, \alpha=3,\, \beta= \alpha^{a} \,\mathrm{mod}\, p
        ;\] inoltre mantiene segreto l'esponente $a=77$.
        \begin{enumerate}
            \item Verificare la correttezza dei dati forniti, in base alle ipotesi del metodo di 
                El Gamal; se $\alpha=3$ non fosse una scelta valida, Bob userà $\alpha\in\{4,\,5\}$ 
                (da verificare). Se nemmeno questa scelta risultasse valida, Bob rinuncia a proseguire 
                (l'esercizio termina); calcolare $\beta$.
            \item Alice estrae il numero casuale segreto (nonce) $k=86$ e spedisce il messaggio $P=100$; 
                calcolare il messaggio cifrato $C=(r,\,t)$.
            \item Bob, per un errore di trasmissione, riceve $C^{\prime}=(r^{\prime},\,t^{\prime})=(44,\,55)$; 
                calcolare il messaggio $P^{\prime}$ decifrato da Bob.
        \end{enumerate}

        \Problem{18/07/2014}
        Bob adotta il sistema di cifratura a chiave pubblica di El Gamal e pubblica i seguenti dati: \[
            p=163,\, \alpha=3,\, \beta= \alpha^{a} \,\mathrm{mod}\, p
        ;\] inoltre mantiene segreto l'esponente $a=128$.
        \begin{enumerate}
            \item Verificare la correttezza dei dati forniti, in base alle ipotesi del metodo di 
                El Gamal; se $\alpha=3$ non fosse una scelta valida, Bob userà $\alpha\in\{4,\,5\}$ 
                (da verificare). Se nemmeno questa scelta risultasse valida, Bob rinuncia a proseguire 
                (l'esercizio termina); calcolare $\beta$.
            \item Alice estrae il numero casuale segreto (nonce) $k=18$ e spedisce il messaggio $P=120$; 
                calcolare il messaggio cifrato $C=(r,\,t)$.
            \item Bob, per un errore di trasmissione, riceve $C^{\prime}=(r^{\prime},\,t^{\prime})=(11,\,22)$; 
                calcolare il messaggio $P^{\prime}$ decifrato da Bob.
        \end{enumerate}

        \Problem{01/09/2014}
        Bob adotta il sistema di cifratura a chiave pubblica di El Gamal e pubblica i seguenti dati: \[
            p=101,\, \alpha=5,\, \beta= \alpha^{a} \,\mathrm{mod}\, p
        ;\] inoltre mantiene segreto l'esponente $a=32$.
        \begin{enumerate}
            \item Verificare la correttezza dei dati forniti, in base alle ipotesi del metodo di 
                El Gamal; se $\alpha=5$ non fosse una scelta valida, Bob userà $\alpha\in\{6,\,7\}$ 
                (da verificare). Se nemmeno questa scelta risultasse valida, Bob rinuncia a proseguire 
                (l'esercizio termina); calcolare $\beta$.
            \item Alice estrae il numero casuale segreto (nonce) $k=64$ e spedisce il messaggio $P_1=50$; 
                calcolare il messaggio cifrato $C_1=(r_1,\,t_1)$.
            \item Alice estrae un nuovo nonce $k$, e usando questo stesso valore spedisce i messaggi 
                $P_2,\,P_3,\,P_4$; Oscar intercetta i seguenti messaggi cifrati, e scopre uno dei messaggi 
                in chiaro: \[
                \begin{array}{ll}
                    C_2=(r_2,\,t_2)=(39,\,5) & P_2=16\\
                    C_3=(r_3,\,t_3)=(39,\,82) & P_3=?\\
                    C_4=(r_4,\,t_4)=(39,\,76) & P_4=?
                \end{array}
                \] Calcolare $P_3$ e $P_4$.
        \end{enumerate}

        \Problem{13/02/2015}
        Bob adotta il sistema di cifratura a chiave pubblica di El Gamal e pubblica i seguenti dati: \[
            p=197,\, \alpha=5,\, \beta= \alpha^{a} \,\mathrm{mod}\, p
        ;\] inoltre mantiene segreto l'esponente $a=128$.
        \begin{enumerate}
            \item Verificare la correttezza dei dati forniti, in base alle ipotesi del metodo di 
                El Gamal; se $\alpha=5$ non fosse una scelta valida, Bob userà $\alpha\in\{6,\,7\}$ 
                (da verificare). Se nemmeno questa scelta risultasse valida, Bob rinuncia a proseguire 
                (l'esercizio termina); calcolare $\beta$.
            \item Alice estrae il numero casuale segreto (nonce) $k=64$ e spedisce il messaggio $P_1=20$; 
                calcolare il messaggio cifrato $C_1=(r_1,\,t_1)$.
            \item Alice estrae un nuovo nonce $k$, e usando questo stesso valore spedisce i messaggi 
                $P_2,\,P_3,\,P_4$; Oscar intercetta i seguenti messaggi cifrati, e scopre uno dei messaggi 
                in chiaro: \[
                \begin{array}{ll}
                    C_2=(r_2,\,t_2)=(138,\,52) & P_2=50\\
                    C_3=(r_3,\,t_3)=(138,\,73) & P_3=?\\
                    C_4=(r_4,\,t_4)=(138,\,168) & P_4=?
                \end{array}
                \] Calcolare $P_3$ e $P_4$.
        \end{enumerate}

        \Problem{23/09/2016}
        Bob adotta il sistema di cifratura a chiave pubblica di El Gamal e pubblica i seguenti dati: \[
            p=127,\, \alpha=7,\, \beta= \alpha^{a} \,\mathrm{mod}\, p
        ;\] inoltre mantiene segreto l'esponente $a=71$.
        \begin{enumerate}
            \item Verificare la correttezza dei dati forniti, in base alle ipotesi del metodo di 
                El Gamal; se $\alpha=7$ non fosse una scelta valida, Bob userà $\alpha\in\{8,\,9\}$ 
                (da verificare). Se nemmeno questa scelta risultasse valida, Bob rinuncia a proseguire 
                (l'esercizio termina); calcolare $\beta$.
            \item Alice estrae il numero casuale segreto (nonce) $k=26$ e spedisce il messaggio $P=100$; 
                calcolare il messaggio cifrato $C=(r,\,t)$.
            \item Bob, per un errore di trasmissione, riceve $C^{\prime}=(r^{\prime},\,t^{\prime})=(72,\,55)$; 
                calcolare il messaggio $P^{\prime}$ decifrato da Bob.
        \end{enumerate}

        \Problem{09/02/2017}
        Bob adotta il sistema di cifratura a chiave pubblica di El Gamal e pubblica i seguenti dati: \[
            p=223,\, \alpha=7,\, \beta= \alpha^{a} \,\mathrm{mod}\, p
        ;\] inoltre mantiene segreto l'esponente $a=129$.
        \begin{enumerate}
            \item Verificare la correttezza dei dati forniti, in base alle ipotesi del metodo di 
                El Gamal; se $\alpha=7$ non fosse una scelta valida, Bob userà $\alpha\in\{6,\,5\}$ 
                (da verificare). Se nemmeno questa scelta risultasse valida, Bob rinuncia a proseguire 
                (l'esercizio termina); calcolare $\beta$.
            \item Alice estrae il numero casuale segreto (nonce) $k=67$ e spedisce il messaggio $P_1=200$; 
                calcolare il messaggio cifrato $C_1=(r_1,\,t_1)$.
            \item Alice estrae un nuovo nonce $k$, e usando questo stesso valore spedisce i messaggi 
                $P_2,\,P_3,\,P_4$; Oscar intercetta i seguenti messaggi cifrati, e scopre uno dei messaggi 
                in chiaro: \[
                \begin{array}{ll}
                    C_2=(r_2,\,t_2)=(129,\,108) & P_2=50\\
                    C_3=(r_3,\,t_3)=(129,\,128) & P_3=?\\
                    C_4=(r_4,\,t_4)=(129,\,100) & P_4=?
                \end{array}
                \] Calcolare $P_3$ e $P_4$.
        \end{enumerate}

        \Problem{27/07/2018}
        Bob adotta il sistema di cifratura a chiave pubblica di El Gamal e pubblica i seguenti dati: \[
            p=89,\, \alpha=7,\, \beta= \alpha^{a} \,\mathrm{mod}\, p
        ;\] inoltre mantiene segreto l'esponente $a=71$.
        \begin{enumerate}
            \item Verificare la correttezza dei dati forniti, in base alle ipotesi del metodo di 
                El Gamal; se $\alpha=7$ non fosse una scelta valida, Bob userà $\alpha\in\{8,\,9\}$ 
                (da verificare). Se nemmeno questa scelta risultasse valida, Bob rinuncia a proseguire 
                (l'esercizio termina); calcolare $\beta$.
            \item Alice estrae il numero casuale segreto (nonce) $k=26$ e spedisce il messaggio $P=50$; 
                calcolare il messaggio cifrato $C=(r,\,t)$.
            \item Bob, per un errore di trasmissione, riceve $C^{\prime}=(r^{\prime},\,t^{\prime})=(21,\,22)$; 
                calcolare il messaggio $P^{\prime}$ decifrato da Bob.
        \end{enumerate}

        \Problem{30/08/2018}
        Bob adotta il sistema di cifratura a chiave pubblica di El Gamal e pubblica i seguenti dati: \[
            p=107,\, \alpha=3,\, \beta= \alpha^{a} \,\mathrm{mod}\, p
        ;\] inoltre mantiene segreto l'esponente $a=40$.
        \begin{enumerate}
            \item Verificare la correttezza dei dati forniti, in base alle ipotesi del metodo di 
                El Gamal; se $\alpha=3$ non fosse una scelta valida, Bob userà $\alpha\in\{4,\,5\}$ 
                (da verificare). Se nemmeno questa scelta risultasse valida, Bob rinuncia a proseguire 
                (l'esercizio termina); calcolare $\beta$.
            \item Alice estrae il numero casuale segreto (nonce) $k=26$ e spedisce il messaggio $P_1=100$; 
                calcolare il messaggio cifrato $C_1=(r_1,\,t_1)$.
            \item Alice estrae un nuovo nonce $k$, e usando questo stesso valore spedisce i messaggi 
                $P_2,\,P_3,\,P_4$; Oscar intercetta i seguenti messaggi cifrati, e scopre uno dei messaggi 
                in chiaro: \[
                \begin{array}{ll}
                    C_2=(r_2,\,t_2)=(18,\,19) & P_2=99\\
                    C_3=(r_3,\,t_3)=(18,\,14) & P_3=?\\
                    C_4=(r_4,\,t_4)=(18,\,28) & P_4=?
                \end{array}
                \] Calcolare $P_3$ e $P_4$.
        \end{enumerate}

        \Problem{11/02/2019}
        Bob adotta il sistema di cifratura a chiave pubblica di El Gamal e pubblica i seguenti dati: \[
            p=107,\, \alpha=6,\, \beta= \alpha^{a} \,\mathrm{mod}\, p
        ;\] inoltre mantiene segreto l'esponente $a=50$.
        \begin{enumerate}
            \item Verificare la correttezza dei dati forniti, in base alle ipotesi del metodo di 
                El Gamal; se $\alpha=6$ non fosse una scelta valida, Bob userà $\alpha\in\{9,\,10\}$ 
                (da verificare). Se nemmeno questa scelta risultasse valida, Bob rinuncia a proseguire 
                (l'esercizio termina); calcolare $\beta$.
            \item Alice estrae il numero casuale segreto (nonce) $k=25$ e spedisce il messaggio $P_1=100$; 
                calcolare il messaggio cifrato $C_1=(r_1,\,t_1)$.
            \item Alice estrae un nuovo nonce $k$, e usando questo stesso valore spedisce i messaggi 
                $P_2,\,P_3,\,P_4$; Oscar intercetta i seguenti messaggi cifrati, e scopre uno dei messaggi 
                in chiaro: \[
                \begin{array}{ll}
                    C_2=(r_2,\,t_2)=(27,\,97) & P_2=50\\
                    C_3=(r_3,\,t_3)=(27,\,95) & P_3=?\\
                    C_4=(r_4,\,t_4)=(27,\,91) & P_4=?
                \end{array}
                \] Calcolare $P_3$ e $P_4$.
        \end{enumerate}

        \Problem{02/07/2019}
        Bob adotta il sistema di cifratura a chiave pubblica di El Gamal e pubblica i seguenti dati: \[
            p=193,\, \alpha=4,\, \beta= \alpha^{a} \,\mathrm{mod}\, p
        ;\] inoltre mantiene segreto l'esponente $a=59$.
        \begin{enumerate}
            \item Verificare la correttezza dei dati forniti, in base alle ipotesi del metodo di 
                El Gamal; se $\alpha=4$ non fosse una scelta valida, Bob userà $\alpha\in\{5,\,6\}$ 
                (da verificare). Se nemmeno questa scelta risultasse valida, Bob rinuncia a proseguire 
                (l'esercizio termina); calcolare $\beta$.
            \item Alice estrae il numero casuale segreto (nonce) $k=67$ e spedisce il messaggio $P_1=200$; 
                calcolare il messaggio cifrato $C_1=(r_1,\,t_1)$.
            \item Alice estrae un nuovo nonce $k$, e usando questo stesso valore spedisce i messaggi 
                $P_2,\,P_3,\,P_4$; Oscar intercetta i seguenti messaggi cifrati, e scopre uno dei messaggi 
                in chiaro: \[
                \begin{array}{ll}
                    C_2=(r_2,\,t_2)=(147,\,183) & P_2=11\\
                    C_3=(r_3,\,t_3)=(147,\,163) & P_3=?\\
                    C_4=(r_4,\,t_4)=(147,\,123) & P_4=?
                \end{array}
                \] Calcolare $P_3$ e $P_4$.
        \end{enumerate}

        \Problem{26/07/2019}
        Bob adotta il sistema di cifratura a chiave pubblica di El Gamal e pubblica i seguenti dati: \[
            p=127,\, \alpha=5,\, \beta= \alpha^{a} \,\mathrm{mod}\, p
        ;\] inoltre mantiene segreto l'esponente $a=64$.
        \begin{enumerate}
            \item Verificare la correttezza dei dati forniti, in base alle ipotesi del metodo di 
                El Gamal; se $\alpha=5$ non fosse una scelta valida, Bob userà $\alpha\in\{6,\,8\}$ 
                (da verificare). Se nemmeno questa scelta risultasse valida, Bob rinuncia a proseguire 
                (l'esercizio termina); calcolare $\beta$.
            \item Alice estrae il numero casuale segreto (nonce) $k=11$ e spedisce il messaggio $P=10$; 
                calcolare il messaggio cifrato $C=(r,\,t)$.
            \item Bob riceve $C^{\prime}=(r^{\prime},\,t^{\prime})=(89,\,117)$. Calcolare il messaggio $P^{\prime}$ 
                decifrato da Bob; per quale valore di $k$ Alice ha calcolato $C^{\prime}=E(P^{\prime})$?
        \end{enumerate}

        \Problem{31/08/2019}
        Bob adotta il sistema di cifratura a chiave pubblica di El Gamal e pubblica i seguenti dati: \[
            p=223,\, \alpha=3,\, \beta= \alpha^{a} \,\mathrm{mod}\, p
        ;\] inoltre mantiene segreto l'esponente $a=70$.
        \begin{enumerate}
            \item Verificare la correttezza dei dati forniti, in base alle ipotesi del metodo di 
                El Gamal; se $\alpha=3$ non fosse una scelta valida, Bob userà $\alpha\in\{4,\,7\}$ 
                (da verificare). Se nemmeno questa scelta risultasse valida, Bob rinuncia a proseguire 
                (l'esercizio termina); calcolare $\beta$.
            \item Alice estrae il numero casuale segreto (nonce) $k=15$ e spedisce il messaggio $P_1=230$; 
                calcolare il messaggio cifrato $C_1=(r_1,\,t_1)$.
            \item Alice estrae un nuovo nonce $k$, e usando questo stesso valore spedisce i messaggi 
                $P_2,\,P_3,\,P_4$; Oscar intercetta i seguenti messaggi cifrati, e scopre uno dei messaggi 
                in chiaro: \[
                \begin{array}{ll}
                    C_2=(r_2,\,t_2)=(27,\,83) & P_2=20\\
                    C_3=(r_3,\,t_3)=(27,\,91) & P_3=?\\
                    C_4=(r_4,\,t_4)=(27,\,200) & P_4=?
                \end{array}
                \] Calcolare $P_3$ e $P_4$.
        \end{enumerate}

    %! TEX root = main.tex
% Capitolo 3

\chapter{Certificati e comunicazione sicura}

    \bigskip
    \section{Certificati X.509}
        \Problem{08/02/2013}
        Una autorità fidata TA rilascia ad Alice il certificato $C_A =$, 
        nel per il quale:
        \begin{itemize}
            \item la cifratura e la firma usano il sistema RSA con $n=221$;
            \item la chiave pubblica di TA è $K_{TA}=35$;
            \item la chiave pubblica di Alice è $K_A=25$;
            \item l'identificativo di Alice vale $A=200$;
            \item la funzione di hash $\mathbb{H}(x,\,y)$ sia definita per $h,\,x,\,y \in \mathbb{Z}_{n}$ come: \[
                    \mathbb{H}(x,\,y) = (x \oplus \text{SL}_3 (y) \oplus \text{SL}_4 (y)) \,\mathrm{mod}\,n
                .\] La funzione $\text{SL}_{k} (h)$ scorre ciclicamente a sinistra di $k$ posizioni il valore 
                binario $h$; infine siano $x,\,y,\,\mathbb{H}(x,\,y)$ parole di 8 bit.
        \end{itemize}
        \begin{enumerate}
            \item Verificare la correttezza dei dati forniti secondo le ipotesi del sistema RSA.
            \item Calcolare il certificato $C_{A}$ rilasciato da TA.
            \item Verificare l'autenticità del certificato $C_{A}$.
        \end{enumerate}

    \bigskip
    \section{Schema di Blom per pre-distribuzione}
        \Problem{11/07/2013}
        L'autorità fidata TA adotta lo schema di Blom per distribuire chiavi simmetriche a tre utenti $A,\,B$ e $C$, 
        e pubblica $p=227$; Gli identificativi pubblici dei tre utenti sono rispettivamente  $r_A=100,\, r_B=101,\, r_C=102$.
        \begin{enumerate}
            \item Per i tre utenti TA sceglie e tiene segreti $a=10,\,b=15,\,c=50$; calcolare le tre chiavi simmetriche 
                $K_{AB},\, K_{AC},\, K_{BC}$ distribuite da TA.
            \item Per i tre utenti, TA sceglie nuovi parametri segreti $a,\,b,\,c$; gli utenti $A$ e $B$ si accordano 
                e si scambiano le informazioni $a_A=22,\, a_B=88,\, b_A=205,\, b_B=77$; calcolare i parametri segreti 
                $a,\,b,\,c$, e poi le tre chiavi simmetriche $K_{AB},\, K_{AC},\, K_{BC}$ distribuite da TA.
        \end{enumerate}

        \Problem{13/09/2013}
        L'autorità fidata TA adotta lo schema di Blom per distribuire chiavi simmetriche a tre utenti $A,\,B$ e $C$, 
        e pubblica $p=997$; Gli identificativi pubblici dei tre utenti sono rispettivamente  $r_A=111,\, r_B=222,\, r_C=333$.
        \begin{enumerate}
            \item Per i tre utenti TA sceglie e tiene segreti $a=501,\,b=955,\,c=370$; calcolare le tre chiavi simmetriche 
                $K_{AB},\, K_{AC},\, K_{BC}$ distribuite da TA.
            \item Per i tre utenti, TA sceglie nuovi parametri segreti $a,\,b,\,c$; gli utenti $A$ e $B$ si accordano 
                e si scambiano le informazioni $a_A=271,\, a_B=117,\, b_A=419,\, b_B=565$; calcolare i parametri segreti 
                $a,\,b,\,c$, e poi le tre chiavi simmetriche $K_{AB},\, K_{AC},\, K_{BC}$ distribuite da TA.
        \end{enumerate}

        \Problem{01/09/2014}
        L'autorità fidata TA adotta lo schema di Blom per distribuire chiavi simmetriche a tre utenti $A,\,B$ e $C$, 
        e pubblica $p=569$; Gli identificativi pubblici dei tre utenti sono rispettivamente  $r_A=25,\, r_B=26,\, r_C=27$.
        \begin{enumerate}
            \item Per i tre utenti TA sceglie e tiene segreti $a=501,\,b=369,\,c=111$; calcolare le tre chiavi simmetriche 
                $K_{AB},\, K_{AC},\, K_{BC}$ distribuite da TA.
            \item Per i tre utenti, TA sceglie nuovi parametri segreti $a,\,b,\,c$; gli utenti $A$ e $B$ si accordano 
                e si scambiano le informazioni $a_A=510,\, a_B=530,\, b_A=201,\, b_B=231$; calcolare i parametri segreti 
                $a,\,b,\,c$, e poi le tre chiavi simmetriche $K_{AB},\, K_{AC},\, K_{BC}$ distribuite da TA.
        \end{enumerate}

        \Problem{23/01/2019}
        L'autorità fidata TA adotta lo schema di Blom per distribuire chiavi simmetriche di sessione 
        $K_{i,\,j}\equiv K_{j,\,i}$ a 100 utenti $U_{k}\,:\,k\in [1,\,\ldots,\,N]$ per la comunicazione tra essi; 
        TA sceglie i valori segreti $a,\,b,\,c$ e pubblica $p$. Un provider fornisce canali sicuri da TA verso 
        ciascun utente, ma a pagamento.
        \begin{enumerate}
            \item Quanti numeri devono essere inviati in tutto da TA, secondo lo schema di Blom?
            \item Se TA generasse centralmente tutte le possibili chiavi di sessione e le inviasse ai rispettivi 
                utenti, quanti numeri dovrebbe inviare in tutto?
            \item Si consideri il caso di tre soli utenti $A,\,b$ e $C$ con identificativi rispettivamente 
                $r_A=10,\, r_B=20,\, r_C=30$; TA sceglie e tiene segreti $a,\,b,\,c$ e pubblica $p=1009$, tuttavia gli utenti 
                $A$ e $B$ si mettono d'accordo e si scambiano le informazioni $a_A=591,\,a_B=73,\,b_A=132,\,b_B=114$. 
                Calcolare i parametri segreti $a,\,b,\,c$, e poi le tre chiavi simmetriche $K_{AB},\, K_{AC},\, K_{BC}$ 
                distribuite da TA.
        \end{enumerate}

        \Problem{31/08/2019}
        L'autorità fidata TA adotta lo schema di Blom per distribuire chiavi simmetriche di sessione 
        $K_{i,\,j}\equiv K_{j,\,i}$ a 1000 utenti $U_{k}\,:\,k\in [1,\,\ldots,\,N]$ per la comunicazione tra essi; 
        TA sceglie i valori segreti $a,\,b,\,c$ e pubblica $p$. Un provider fornisce canali sicuri da TA verso 
        ciascun utente, ma a pagamento.
        \begin{enumerate}
            \item Quanti numeri devono essere inviati in tutto da TA, secondo lo schema di Blom?
            \item Se TA generasse centralmente tutte le possibili chiavi di sessione e le inviasse ai rispettivi 
                utenti, quanti numeri dovrebbe inviare in tutto?
            \item Si consideri il caso di tre soli utenti $A,\,b$ e $C$ con identificativi rispettivamente 
                $r_A=100,\, r_B=200,\, r_C=30$; TA sceglie e tiene segreti $a,\,b,\,c$ e pubblica $p=1009$, 
                tuttavia gli utenti $A$ e $B$ si mettono d'accordo e si scambiano le informazioni 
                $a_A=929,\,a_B=838,\,b_A=173,\,b_B=276$. Calcolare i parametri segreti $a,\,b,\,c$, e poi le 
                tre chiavi simmetriche $K_{AB},\, K_{AC},\, K_{BC}$ distribuite da TA.
        \end{enumerate}

    \include{chap4}
    %! TEX root = main.tex
% Capitolo 5

\chapter{Sequenze binarie pseudo-casuali}

    \bigskip
    \section{Generatore di Blum, Blum e Shub}
        \Problem{27/02/2013}
        Ricavare la sequenza binaria pseudo\nbdash casuale generata dall'algoritmo \emph{Blum-Blum-Shub} 
        per $p=7,\, q=11,\, x=5$ e determinarne il periodo.

        \Problem{01/09/2014}
        Ricavare la sequenza binaria pseudo\nbdash casuale generata dall'algoritmo \emph{Blum-Blum-Shub} 
        per $p=19,\, q=31,\, x=9$ e determinarne il periodo.

        \Problem{17/07/2015}
        Ricavare la sequenza binaria pseudo\nbdash casuale generata dall'algoritmo \emph{Blum-Blum-Shub} 
        per $p=23,\, q=31,\, x=5$ e determinarne il periodo.

        \Problem{04/07/2016}
        \begin{enumerate}
            \item Ricavare la sequenza binaria pseudo\nbdash casuale generata dall'algoritmo \emph{Blum-Blum-Shub} 
                per $p=19,\, q=31,\, x=3$ e determinarne il periodo.
            \item In base alla teoria, quali sono i valori possibili che può assumere il periodo $P = \pi(x_0)$ del 
                generatore ottenuto al punto precedente, per valori arbitrari del seme $x_0 =x^2 \in\mathbb{Z}_n$?\\
                Si ricorda che $\pi(x_0)$ divide $\lambda(\lambda(n))$, dove 
                $\lambda(n)=\mathrm{mcm}\left(\{\varphi(p^{a_i}_i)\}\right)$ è la funzione di Carmichael.
        \end{enumerate}

        \Problem{09/02/2017}
        \begin{enumerate}
            \item Ricavare la sequenza binaria pseudo\nbdash casuale generata dall'algoritmo \emph{Blum-Blum-Shub} 
                per $p=23,\, q=43,\, x=6$ e determinarne il periodo.
            \item In base alla teoria, quali sono i valori possibili che può assumere il periodo $P = \pi(x_0)$ del 
                generatore ottenuto al punto precedente, per valori arbitrari del seme $x_0 =x^2 \in\mathbb{Z}_n$?\\
                Si ricorda che $\pi(x_0)$ divide $\lambda(\lambda(n))$, dove 
                $\lambda(n)=\mathrm{mcm}\left(\{\varphi(p^{a_i}_i)\}\right)$ è la funzione di Carmichael.
        \end{enumerate}

        \Problem{09/02/2017}
        \begin{enumerate}
            \item Ricavare la sequenza binaria pseudo\nbdash casuale generata dall'algoritmo \emph{Blum-Blum-Shub} 
                per $p=23,\, q=47,\, x=48$ e determinarne il periodo.
            \item In base alla teoria, quali sono i valori possibili che può assumere il periodo $P = \pi(x_0)$ del 
                generatore ottenuto al punto precedente, per valori arbitrari del seme $x_0 =x^2 \in\mathbb{Z}_n$?\\
                Si ricorda che $\pi(x_0)$ divide $\lambda(\lambda(n))$, dove 
                $\lambda(n)=\mathrm{mcm}\left(\{\varphi(p^{a_i}_i)\}\right)$ è la funzione di Carmichael.
        \end{enumerate}

    \bigskip
    \section{Scrambler additivo}
        \Problem{11/07/2013}
        Si disegni lo schema di uno scrambler additivo avente polinomio caratteristico $1+x^2 +x^3 =x^5$. Quale 
        sarà il massimo periodo della sequenza PRBS generata, avendo tutti ``0'' in ingresso?

        \Problem{25/09/2013}
        \begin{enumerate}
            \item Si disegni la figura di uno scrambler additivo avente polinomio caratteristico $P(x)= 1+x +x^2 +x^4$. 
                Si indichi la sequenza binaria in ingresso con $\{I_k\}$, quella in uscita con $\{U_k\}$ e quella 
                pseudo\nbdash casuale generata con $\{R_k\}$.
            \item Si inizializzi lo scrambler con  ``1'' negli elementi di ritardo $D_1$ e $D_2$ e con ``0'' negli elementi 
                $D_3$ e $D_4$; lo si alimenti con una sequenza in ingresso di soli ``1''. Ricavare la sequenza $\{U_k\}$ 
                all'uscita, evidenziando la sua periodicità: qual è il periodo di tale sequenza?
            \item Verificare che il polinomio $P(x)$ sia irriducibile.
        \end{enumerate}

        \Problem{23/09/2016}
        \begin{enumerate}
            \item Si disegni la figura di uno scrambler additivo avente polinomio caratteristico $P(x)= 1+x +x^3 +x^4$. 
                Si indichi la sequenza binaria in ingresso con $\{I_k\}$, quella in uscita con $\{U_k\}$ e quella 
                pseudo\nbdash casuale generata con $\{R_k\}$.
            \item Si inizializzi lo scrambler con  ``1'' negli elementi di ritardo $D_1$ e con ``0'' negli elementi 
                $D_2$, $D_3$ e $D_4$; lo si alimenti con una sequenza in ingresso di soli ``1''. Ricavare la 
                sequenza $\{U_k\}$ all'uscita, evidenziando la sua periodicità: qual è il periodo di tale sequenza?
            \item Verificare che il polinomio $P(x)$ sia irriducibile.
        \end{enumerate}

    \bigskip
    \section{Scrambler autosincronizzante}
        \Problem{18/07/2014}
        \begin{enumerate}
            \item Si disegni la figura di uno scrambler autosincronizzante avente polinomio caratteristico 
                $P(x)= 1 +x^2 +x^4$. Si indichi la sequenza binaria in ingresso con $\{I_k\}$ e quella in 
                uscita con $\{U_k\}$.
            \item Si inizializzi lo scrambler con  ``1'' negli elementi di ritardo $D_1$ e con ``0'' negli elementi 
                $D_2$, $D_3$ e $D_4$; lo si alimenti con una sequenza in ingresso di soli ``1''. Ricavare la 
                sequenza $\{U_k\}$ all'uscita, evidenziando la sua periodicità: qual è il periodo di tale sequenza?
            \item Verificare che il polinomio $P(x)$ sia irriducibile.
        \end{enumerate}

        \Problem{03/07/2015}
        \begin{enumerate}
            \item Si disegni la figura di uno scrambler autosincronizzante avente polinomio caratteristico 
                $P(x)= 1 +x^4$. Si indichi la sequenza binaria in ingresso con $\{I_k\}$ e quella in 
                uscita con $\{U_k\}$.
            \item Si inizializzi lo scrambler con  ``1'' negli elementi di ritardo $D_i$; lo si alimenti con una 
                sequenza in ingresso di soli ``1''. Ricavare la sequenza $\{U_k\}$ all'uscita, evidenziando la sua 
                periodicità: qual è il periodo di tale sequenza?
            \item Verificare che il polinomio $P(x)$ sia irriducibile.
        \end{enumerate}

        \Problem{30/08/2018}
        \begin{enumerate}
            \item Si disegni la figura di uno scrambler autosincronizzante avente polinomio caratteristico 
                $P(x)= 1 +x +x^3 +x^4$, alimentato da una sequenza di soli ``0'' in ingresso e usato come generatore PRBS. 
                Si indichi la sequenza binaria in ingresso con $\{I_k\}=\{0\}$ e quella in uscita con $\{R_k\}$.
            \item Si inizializzi lo scrambler con $D_i(i=1,\,2,\,3,\,4)$ con $\{1,\,0,\,0,\,0\}$ al passo $k=0$. 
                Ricavare la sequenza $\{R_k\}$ all'uscita, evidenziando la sua periodicità: qual è il periodo di 
                tale sequenza?
            \item Verificare che il polinomio $P(x)$ sia irriducibile. Se lo fosse, quali sarebbero i valori 
                possibili per il periodo?
        \end{enumerate}

        \Problem{11/02/2019}
        \begin{enumerate}
            \item Si disegni la figura di uno scrambler autosincronizzante avente polinomio caratteristico 
                $P(x)= 1 +x +x^2 +x^4$, alimentato da una sequenza di soli ``0'' in ingresso e usato come generatore PRBS. 
                Si indichi la sequenza binaria in ingresso con $\{I_k\}=\{0\}$ e quella in uscita con $\{R_k\}$.
            \item Si inizializzi lo scrambler con $D_i(i=1,\,2,\,3,\,4)$ con $\{1,\,1,\,0,\,0\}$ al passo $k=0$. 
                Ricavare la sequenza $\{R_k\}$ all'uscita, evidenziando la sua periodicità: qual è il periodo di 
                tale sequenza?
            \item Verificare che il polinomio $P(x)$ sia irriducibile. Se lo fosse, quali sarebbero i valori 
                possibili per il periodo?
        \end{enumerate}

        \Problem{02/07/2019}
        \begin{enumerate}
            \item Si disegni lo schema di un generatore PRBS basato su registro a scorrimento LFSR, realizzato con uno 
                scrambler autosincronizzante avente polinomio caratteristico $P(x)= 1 +x^2 +x^4$, alimentato da una 
                sequenza di soli ``0'' in ingresso. Si indichi la sequenza binaria in ingresso con $\{I_k\}=\{0\}$ e 
                quella in uscita con $\{R_k\}$.
            \item Si inizializzi lo scrambler con $D_i(i=1,\,2,\,3,\,4)$ con $\{0,\,1,\,0,\,0\}$ al passo $k=0$. 
                Ricavare la sequenza $\{R_k\}$ all'uscita, evidenziando la sua periodicità: qual è il periodo di 
                tale sequenza?
            \item Verificare che il polinomio $P(x)$ sia irriducibile. Se lo fosse, quali sarebbero i valori 
                possibili per il periodo?
        \end{enumerate}


%     \part{Soluzioni}
%     %! TEX root = main.tex
% Capitolo 1

\chapter{Firma digitale}

    \bigskip
    \section{Firma RSA}
        \ASolution{27/02/2013}

    \bigskip
    \section{Firma Cieca}
        \ASolution{25/09/2013}

        \ASolution{18/07/2014}

        \ASolution{03/07/2015}

        \ASolution{21/07/2016}

        \ASolution{09/02/2017}

        \ASolution{30/08/2018}

    \bigskip
    \section{Firma di El Gamal}
        \ASolution{08/02/2013}

        \ASolution{11/07/2013}

        \ASolution{13/09/2013}

        \ASolution{03/03/2014}

        \ASolution{04/07/2014}

        \ASolution{17/09/2014}

        \ASolution{13/02/2015}

        \ASolution{02/03/2015}

        \ASolution{17/07/2015}

        \ASolution{04/07/2016}

        \ASolution{21/07/2016}

        \ASolution{09/09/2016}

        \ASolution{06/07/2018}

        \ASolution{27/07/2018}

        \ASolution{23/01/2019}

        \ASolution{23/01/2019}

        \ASolution{11/02/2019}

        \ASolution{02/07/2019}

        \ASolution{26/07/2019}

        \ASolution{31/08/2019}

%     %! TEX root = main.tex

\chapter{Sequenze binarie pseudo-casuali}

    \bigskip
    \section{Generatore di Blum, Blum e Shub}
        \ASolution{27/02/2013}

        \ASolution{01/09/2014}

        \ASolution{17/07/2015}

        \ASolution{04/07/2016}

        \ASolution{09/02/2017}

        \ASolution{09/02/2017}

    \bigskip
    \section{Scrambler additivo}
        \ASolution{11/07/2013}

        \ASolution{25/09/2013}

        \ASolution{23/09/2016}

    \bigskip
    \section{Scrambler autosincronizzante}
        \ASolution{18/07/2014}

        \ASolution{03/07/2015}

        \ASolution{30/08/2018}

        \ASolution{11/02/2019}

        \ASolution{02/07/2019}

%     \include{chap3-s}
%     %! TEX root = main.tex
% Capitolo 4

\chapter{Aritmetica modulare}

    \bigskip
    \section{Potenze}
        \ASolution{11/07/2013}

        \ASolution{04/07/2014}

    \bigskip
    \section{Radici}
        \ASolution{27/02/2013}

        \ASolution{18/07/2014}

        \ASolution{31/08/2019}

    \bigskip
    \section{Inverso}
        \ASolution{08/02/2013}

        \ASolution{11/07/2013}

        \ASolution{04/07/2014}

        \ASolution{17/09/2014}

        \ASolution{03/07/2015}

    \bigskip
    \section{Residui quadratici}
        \ASolution{08/02/2013}

        \ASolution{27/02/2013}

        \ASolution{14/02/2014}

        \ASolution{17/09/2014}

        \ASolution{02/03/2015}

        \ASolution{03/07/2015}

        \ASolution{09/02/2017}

    \bigskip
    \section{Logaritmo discreto}
        \ASolution{08/02/2013}

        \ASolution{03/03/2014}

        \ASolution{17/07/2015}

        \ASolution{23/09/2016}

        \ASolution{06/07/2018}

    \bigskip
    \section{Fattorizzazione}
        \ASolution{04/07/2016}

        \ASolution{21/07/2016}

        \ASolution{09/02/2017}

        \ASolution{23/09/2016}

        \ASolution{30/08/2018}

        \ASolution{11/02/2019}

        \ASolution{26/07/2019}

        \ASolution{31/08/2019}

%     %! TEX root = main.tex

\chapter{Certificati e comunicazione sicura}

    \bigskip
    \section{Certificati X.509}
        \ASolution{08/02/2013}

    \bigskip
    \section{Schema di Blom per pre-distribuzione}
        \ASolution{11/07/2013}

        \ASolution{13/09/2013}

        \ASolution{01/09/2014}

        \ASolution{23/01/2019}

        \ASolution{31/08/2019}


%    \part{Quesiti}
%    %! TEX root = main.tex

\chapter{Crittografia}

    \bigskip
    \section{Teoria dei numeri}
        \Problem{01/01/2000}
        Pollo arrosto

    \bigskip
    \section{Cifrari elementari}
        \Problem{01/01/2000}
        Pollo arrosto

    \bigskip
    \section{Cifrario DES}
        \Problem{01/01/2000}
        Pollo arrosto

    \bigskip
    \section{Cifrario AES}
        \Problem{01/01/2000}
        Pollo arrosto

    \bigskip
    \section{Successioni Pseudo-casuali}
        \Problem{01/01/2000}
        Pollo arrosto

    \bigskip
    \section{RSA}
        \Problem{01/01/2000}
        Pollo arrosto

    \bigskip
    \section{Firma digitale}
        \Problem{01/01/2000}
        Pollo arrosto

%    %! TEX root = main.tex

\chapter{Comunicazione sicura}

    \bigskip
    \section{Instaurazione e distribuzione della chiave}
        \Problem{01/01/2000}
        Pollo arrosto

    \bigskip
    \section{Autenticazione e autorizzazione}
        \Problem{01/01/2000}
        Pollo arrosto

    \bigskip
    \section{Certificati e protocolli di rete}
        \Problem{01/01/2000}
        Pollo arrosto


\end{document}
