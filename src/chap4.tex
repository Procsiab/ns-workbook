%! TEX root = main.tex
% Capitolo 4

\chapter{Aritmetica modulare}

    \bigskip
    \section{Potenze}
        \Problem{11/07/2013}
        Calcolare $8751423^{65536} \mod 131072$

        \Problem{04/07/2014}
        Siano noti $n=p\cdot q =314869$ prodotto di due primi distinti $p$ e $q$, e $\varphi(n)=313740$: calcolare 
        $p$ e $q$.

    \bigskip
    \section{Radici}
        \Problem{27/02/2013}
        L'equazione $\cgm{x^2}{5}{239}$ ha soluzione? Se la risposta è affermativa, calcolarne 
        le radici, altrimenti risolvere $\cgm{x^2}{-5}{239}$.

        \Problem{18/07/2014}
        L'equazione $\cgm{x^2}{25}{223}$ ha soluzione? Se la risposta è affermativa, calcolarne 
        le radici, altrimenti risolvere $\cgm{x^2}{-25}{223}$.

        \Problem{31/08/2019}
        L'equazione $\cgm{x^2}{11}{359}$ ha soluzione? Se la risposta è affermativa, calcolarne 
        le radici, altrimenti risolvere $\cgm{x^2}{-11}{359}$.

    \bigskip
    \section{Inverso}
        \Problem{08/02/2013}
        Calcolare l'inverso $a^{-1}$ di $\cgm{a}{10}{999}$ applicando il \emph{teorema di Eulero} 
        e quindi l'algoritmo \emph{square \& multiply} per il calcolo della potenza.

        \Problem{11/07/2013}
        Calcolare $500^{-1} \mod 977$ per mezzo dell'\emph{algoritmo di Euclide esteso}.

        \Problem{04/07/2014}
        Applicando il \emph{teorema cinese del resto}, trovare la congruenza equivalente a \[
        \begin{cases}
            \cgm{x}{6}{15} \\
            \cgm{x}{15}{16}
        \end{cases}
        ,\] verificando le ipotesi di questo teorema.

        \Problem{17/09/2014}
        Calcolare $23^{-1} \mod 317$ per mezzo dell'\emph{algoritmo di Euclide esteso}.

        \Problem{03/07/2015}
        Calcolare $3\cdot 55^{-1} \mod 317$ per mezzo dell'\emph{algoritmo di Euclide esteso}.

    \bigskip
    \section{Residui quadratici}
        \Problem{08/02/2013}
        Trovare ed elencare i residui quadratici $a_q$ dell'insieme $\mathbb{Z}^{*}_p$ per $p=17$, ovvero gli 
        elementi tali che $\forall a\in \mathbb{Z}^{*}_p \,:\, \cgm{a}{(\pm b)^2}{p} \land b\in \mathbb{Z}^{*}_p$.

        \Problem{27/02/2013}
        \begin{enumerate}
            \item Quali sono gli elementi primitivi $\alpha$ dell'insieme $\mathbb{Z}^{*}_{17}$?
            \item Quanti sono gli elementi primitivi di $\mathbb{Z}^{*}_{17}$?
            \item Trovare ed elencare in ordine gli elementi primitivi di $\mathbb{Z}^{*}_{17}$.
            \item Qual è l'ordine dell'elemento $\alpha =11$?
            \item Qual è l'ordine dell'elemento $\alpha =13$?
        \end{enumerate}

        \Problem{14/02/2014}
        \begin{enumerate}
            \item Trovare ed elencare in ordine crescente gli elementi primitivi di $\mathbb{Z}^{*}_{19}$
            \item Qual è l'ordine dell'elemento $\alpha =11$?
            \item Qual è l'ordine dell'elemento $\alpha =13$?
            \item Trovare ed elencare in ordine crescente i residui quadratici di $\mathbb{Z}^{*}_{19}$, ovvero gli 
                elementi tali che $\forall a\in \mathbb{Z}^{*}_{19} \,:\, \cgm{a}{(\pm b)^2}{19} \land 
                b\in \mathbb{Z}^{*}_{19}$.
        \end{enumerate}

        \Problem{17/09/2014}
        \begin{enumerate}
            \item Cos'è un elemento primitivo $\alpha \in \mathbb{Z}^{*}_p$?
            \item Qual è l'ordine dell'elemento $\alpha =11$? È un elemento primitivo di $\mathbb{Z}^{*}_{23}$?
            \item Qual è l'ordine dell'elemento $\alpha =13$? È un elemento primitivo di $\mathbb{Z}^{*}_{23}$?
            \item Trovare ed elencare in ordine crescente i residui quadratici di $\mathbb{Z}^{*}_{23}$, ovvero gli 
                elementi tali che $\forall a\in \mathbb{Z}^{*}_{23} \,:\, \cgm{a}{(\pm b)^2}{23} \land 
                b\in \mathbb{Z}^{*}_{23}$.
        \end{enumerate}

        \Problem{02/03/2015}
        \begin{enumerate}
            \item Cos'è un residuo quadratico dell'insieme $\mathbb{Z}^{*}_p$?
            \item Dopo aver calcolato tutti i residui quadratici di $\mathbb{Z}^{*}_{19}$, dire quali sono le radici 
                quadrate di  $-2 \mod 19$.
        \end{enumerate}

        \Problem{03/07/2015}
        \begin{enumerate}
            \item Cos'è un elemento primitivo $\alpha \in \mathbb{Z}^{*}_p$?
            \item Trovare ed elencare in ordine crescente i primi quattro elementi primitivi di $\mathbb{Z}^{*}_{31}$.
            \item Qual è l'ordine dell'elemento $\alpha =6$?
        \end{enumerate}

        \Problem{09/02/2017}
        \begin{enumerate}
            \item Cos'è un elemento primitivo $\alpha \in \mathbb{Z}^{*}_p$?
            \item Quanti sono gli elementi primitivi di $\mathbb{Z}^{*}_{61}$?
            \item Qual è l'ordine dell'elemento $\alpha =3$?
        \end{enumerate}

    \bigskip
    \section{Logaritmo discreto}
        \Problem{08/02/2013}
        Calcolare il logaritmo discreto, soluzione dell'equazione $\cgm{\alpha^x}{\beta}{p}$ per 
        $p=23,\, \alpha =5,\, \beta =21$, applicando l'algoritmo \emph{baby step, giant step}; verificare per 
        prima cosa se esiste con certezza una soluzione.

        \Problem{03/03/2014}
        Trovare le soluzioni dell'equazione $\cgm{13^x}{2}{19}$.

        \Problem{17/07/2015}
        Calcolare il logaritmo discreto, soluzione dell'equazione $\cgm{\alpha^x}{\beta}{p}$ per 
        $p=89,\, \alpha =7,\, \beta =53$, applicando l'algoritmo \emph{baby step, giant step}; verificare per 
        prima cosa se esiste con certezza una soluzione.

        \Problem{23/09/2016}
        Calcolare il logaritmo discreto, soluzione dell'equazione $\cgm{\alpha^x}{\beta}{p}$ per 
        $p=103,\, \alpha =12,\, \beta =20$, applicando l'algoritmo \emph{baby step, giant step}; verificare per prima 
        cosa se esiste con certezza una soluzione. Per quanti valori di $b\in \mathbb{Z}^{*}_p$ l'equazione 
        ammette soluzione?

        \Problem{06/07/2018}
        Calcolare il logaritmo discreto, soluzione dell'equazione $\cgm{\alpha^x}{\beta}{p}$ per 
        $p=139,\, \alpha =3,\, \beta =40$, applicando l'algoritmo \emph{baby step, giant step}; verificare per prima 
        cosa se esiste con certezza una soluzione. Per quanti valori di $b\in \mathbb{Z}^{*}_p$ l'equazione 
        ammette soluzione?

    \bigskip
    \section{Fattorizzazione}
        \Problem{04/07/2016}
        Trovare i fattori primi di $n=20687$ attraverso l'\emph{algoritmo di fattorizzazione $p-1$ di Pollard} con 
        base $a=2$.

        \Problem{21/07/2016}
        Trovare i fattori primi di $n=1083617$ attraverso il \emph{metodo di fattorizzazione di Fermat}.

        \Problem{09/02/2017}
        Trovare i fattori primi di $n=14803$ attraverso l'\emph{algoritmo di fattorizzazione $p-1$ di Pollard} con 
        base $a=2$.

        \Problem{23/09/2016}
        Trovare i fattori primi di $n=2047697$ attraverso il \emph{metodo di fattorizzazione di Fermat}.

        \Problem{30/08/2018}
        Trovare i fattori primi di $n=38191$ attraverso l'\emph{algoritmo di fattorizzazione $p-1$ di Pollard} con 
        base $a=2$.

        \Problem{11/02/2019}
        Trovare i fattori primi di $n=13843$ attraverso l'\emph{algoritmo di fattorizzazione $p-1$ di Pollard} con 
        base $a=2$.

        \Problem{26/07/2019}
        Trovare i fattori primi di $n=1734389$ attraverso il \emph{metodo di fattorizzazione di Fermat}.

        \Problem{31/08/2019}
        Trovare i fattori primi di $n=19549$ attraverso l'\emph{algoritmo di fattorizzazione $p-1$ di Pollard} con 
        base $a=2$.
