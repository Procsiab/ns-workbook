%! TEX root = main.tex
% Capitolo 3

\chapter{Certificati e comunicazione sicura}

    \bigskip
    \section{Certificati X.509}
        \Problem{08/02/2013}
        Una autorità fidata TA rilascia ad Alice il certificato $C_A =$, 
        nel per il quale:
        \begin{itemize}
            \item la cifratura e la firma usano il sistema RSA con $n=221$;
            \item la chiave pubblica di TA è $K_{TA}=35$;
            \item la chiave pubblica di Alice è $K_A=25$;
            \item l'identificativo di Alice vale $A=200$;
            \item la funzione di hash $\mathbb{H}(x,\,y)$ sia definita per $h,\,x,\,y \in \mathbb{Z}_{n}$ come: \[
                    \mathbb{H}(x,\,y) = (x \oplus \text{SL}_3 (y) \oplus \text{SL}_4 (y)) \,\mathrm{mod}\,n
                .\] La funzione $\text{SL}_{k} (h)$ scorre ciclicamente a sinistra di $k$ posizioni il valore 
                binario $h$; infine siano $x,\,y,\,\mathbb{H}(x,\,y)$ parole di 8 bit.
        \end{itemize}
        \begin{enumerate}
            \item Verificare la correttezza dei dati forniti secondo le ipotesi del sistema RSA.
            \item Calcolare il certificato $C_{A}$ rilasciato da TA.
            \item Verificare l'autenticità del certificato $C_{A}$.
        \end{enumerate}

    \bigskip
    \section{Schema di Blom per pre-distribuzione}
        \Problem{11/07/2013}
        L'autorità fidata TA adotta lo schema di Blom per distribuire chiavi simmetriche a tre utenti $A,\,B$ e $C$, 
        e pubblica $p=227$; Gli identificativi pubblici dei tre utenti sono rispettivamente  $r_A=100,\, r_B=101,\, r_C=102$.
        \begin{enumerate}
            \item Per i tre utenti TA sceglie e tiene segreti $a=10,\,b=15,\,c=50$; calcolare le tre chiavi simmetriche 
                $K_{AB},\, K_{AC},\, K_{BC}$ distribuite da TA.
            \item Per i tre utenti, TA sceglie nuovi parametri segreti $a,\,b,\,c$; gli utenti $A$ e $B$ si accordano 
                e si scambiano le informazioni $a_A=22,\, a_B=88,\, b_A=205,\, b_B=77$; calcolare i parametri segreti 
                $a,\,b,\,c$, e poi le tre chiavi simmetriche $K_{AB},\, K_{AC},\, K_{BC}$ distribuite da TA.
        \end{enumerate}

        \Problem{13/09/2013}
        L'autorità fidata TA adotta lo schema di Blom per distribuire chiavi simmetriche a tre utenti $A,\,B$ e $C$, 
        e pubblica $p=997$; Gli identificativi pubblici dei tre utenti sono rispettivamente  $r_A=111,\, r_B=222,\, r_C=333$.
        \begin{enumerate}
            \item Per i tre utenti TA sceglie e tiene segreti $a=501,\,b=955,\,c=370$; calcolare le tre chiavi simmetriche 
                $K_{AB},\, K_{AC},\, K_{BC}$ distribuite da TA.
            \item Per i tre utenti, TA sceglie nuovi parametri segreti $a,\,b,\,c$; gli utenti $A$ e $B$ si accordano 
                e si scambiano le informazioni $a_A=271,\, a_B=117,\, b_A=419,\, b_B=565$; calcolare i parametri segreti 
                $a,\,b,\,c$, e poi le tre chiavi simmetriche $K_{AB},\, K_{AC},\, K_{BC}$ distribuite da TA.
        \end{enumerate}

        \Problem{01/09/2014}
        L'autorità fidata TA adotta lo schema di Blom per distribuire chiavi simmetriche a tre utenti $A,\,B$ e $C$, 
        e pubblica $p=569$; Gli identificativi pubblici dei tre utenti sono rispettivamente  $r_A=25,\, r_B=26,\, r_C=27$.
        \begin{enumerate}
            \item Per i tre utenti TA sceglie e tiene segreti $a=501,\,b=369,\,c=111$; calcolare le tre chiavi simmetriche 
                $K_{AB},\, K_{AC},\, K_{BC}$ distribuite da TA.
            \item Per i tre utenti, TA sceglie nuovi parametri segreti $a,\,b,\,c$; gli utenti $A$ e $B$ si accordano 
                e si scambiano le informazioni $a_A=510,\, a_B=530,\, b_A=201,\, b_B=231$; calcolare i parametri segreti 
                $a,\,b,\,c$, e poi le tre chiavi simmetriche $K_{AB},\, K_{AC},\, K_{BC}$ distribuite da TA.
        \end{enumerate}

        \Problem{23/01/2019}
        L'autorità fidata TA adotta lo schema di Blom per distribuire chiavi simmetriche di sessione 
        $K_{i,\,j}\equiv K_{j,\,i}$ a 100 utenti $U_{k}\,:\,k\in [1,\,\ldots,\,N]$ per la comunicazione tra essi; 
        TA sceglie i valori segreti $a,\,b,\,c$ e pubblica $p$. Un provider fornisce canali sicuri da TA verso 
        ciascun utente, ma a pagamento.
        \begin{enumerate}
            \item Quanti numeri devono essere inviati in tutto da TA, secondo lo schema di Blom?
            \item Se TA generasse centralmente tutte le possibili chiavi di sessione e le inviasse ai rispettivi 
                utenti, quanti numeri dovrebbe inviare in tutto?
            \item Si consideri il caso di tre soli utenti $A,\,b$ e $C$ con identificativi rispettivamente 
                $r_A=10,\, r_B=20,\, r_C=30$; TA sceglie e tiene segreti $a,\,b,\,c$ e pubblica $p=1009$, tuttavia gli utenti 
                $A$ e $B$ si mettono d'accordo e si scambiano le informazioni $a_A=591,\,a_B=73,\,b_A=132,\,b_B=114$. 
                Calcolare i parametri segreti $a,\,b,\,c$, e poi le tre chiavi simmetriche $K_{AB},\, K_{AC},\, K_{BC}$ 
                distribuite da TA.
        \end{enumerate}

        \Problem{31/08/2019}
        L'autorità fidata TA adotta lo schema di Blom per distribuire chiavi simmetriche di sessione 
        $K_{i,\,j}\equiv K_{j,\,i}$ a 1000 utenti $U_{k}\,:\,k\in [1,\,\ldots,\,N]$ per la comunicazione tra essi; 
        TA sceglie i valori segreti $a,\,b,\,c$ e pubblica $p$. Un provider fornisce canali sicuri da TA verso 
        ciascun utente, ma a pagamento.
        \begin{enumerate}
            \item Quanti numeri devono essere inviati in tutto da TA, secondo lo schema di Blom?
            \item Se TA generasse centralmente tutte le possibili chiavi di sessione e le inviasse ai rispettivi 
                utenti, quanti numeri dovrebbe inviare in tutto?
            \item Si consideri il caso di tre soli utenti $A,\,b$ e $C$ con identificativi rispettivamente 
                $r_A=100,\, r_B=200,\, r_C=30$; TA sceglie e tiene segreti $a,\,b,\,c$ e pubblica $p=1009$, 
                tuttavia gli utenti $A$ e $B$ si mettono d'accordo e si scambiano le informazioni 
                $a_A=929,\,a_B=838,\,b_A=173,\,b_B=276$. Calcolare i parametri segreti $a,\,b,\,c$, e poi le 
                tre chiavi simmetriche $K_{AB},\, K_{AC},\, K_{BC}$ distribuite da TA.
        \end{enumerate}
