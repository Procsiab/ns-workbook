%! TEX root = main.tex

\chapter{Sequenze binarie pseudo-casuali}

    \bigskip
    \section{Generatore di Blum, Blum e Shub}
        \Problem{27/02/2013}
        Ricavare la sequenza binaria pseudo\nbdash casuale generata dall'algoritmo \emph{Blum-Blum-Shub} 
        per $p=7,\, q=11,\, x=5$ e determinarne il periodo.

        \Problem{01/09/2014}
        Ricavare la sequenza binaria pseudo\nbdash casuale generata dall'algoritmo \emph{Blum-Blum-Shub} 
        per $p=19,\, q=31,\, x=9$ e determinarne il periodo.

        \Problem{17/07/2015}
        Ricavare la sequenza binaria pseudo\nbdash casuale generata dall'algoritmo \emph{Blum-Blum-Shub} 
        per $p=23,\, q=31,\, x=5$ e determinarne il periodo.

        \Problem{04/07/2016}
        \begin{enumerate}
            \item Ricavare la sequenza binaria pseudo\nbdash casuale generata dall'algoritmo \emph{Blum-Blum-Shub} 
                per $p=19,\, q=31,\, x=3$ e determinarne il periodo.
            \item In base alla teoria, quali sono i valori possibili che può assumere il periodo $P = \pi(x_0)$ del 
                generatore ottenuto al punto precedente, per valori arbitrari del seme $x_0 =x^2 \in\mathbb{Z}_n$?\\
                Si ricorda che $\pi(x_0)$ divide $\lambda(\lambda(n))$, dove 
                $\lambda(n)=\mathrm{mcm}\left(\{\varphi(p^{a_i}_i)\}\right)$ è la funzione di Carmichael.
        \end{enumerate}

        \Problem{09/02/2017}
        \begin{enumerate}
            \item Ricavare la sequenza binaria pseudo\nbdash casuale generata dall'algoritmo \emph{Blum-Blum-Shub} 
                per $p=23,\, q=43,\, x=6$ e determinarne il periodo.
            \item In base alla teoria, quali sono i valori possibili che può assumere il periodo $P = \pi(x_0)$ del 
                generatore ottenuto al punto precedente, per valori arbitrari del seme $x_0 =x^2 \in\mathbb{Z}_n$?\\
                Si ricorda che $\pi(x_0)$ divide $\lambda(\lambda(n))$, dove 
                $\lambda(n)=\mathrm{mcm}\left(\{\varphi(p^{a_i}_i)\}\right)$ è la funzione di Carmichael.
        \end{enumerate}

        \Problem{09/02/2017}
        \begin{enumerate}
            \item Ricavare la sequenza binaria pseudo\nbdash casuale generata dall'algoritmo \emph{Blum-Blum-Shub} 
                per $p=23,\, q=47,\, x=48$ e determinarne il periodo.
            \item In base alla teoria, quali sono i valori possibili che può assumere il periodo $P = \pi(x_0)$ del 
                generatore ottenuto al punto precedente, per valori arbitrari del seme $x_0 =x^2 \in\mathbb{Z}_n$?\\
                Si ricorda che $\pi(x_0)$ divide $\lambda(\lambda(n))$, dove 
                $\lambda(n)=\mathrm{mcm}\left(\{\varphi(p^{a_i}_i)\}\right)$ è la funzione di Carmichael.
        \end{enumerate}

    \bigskip
    \section{Scrambler additivo}
        \Problem{11/07/2013}
        Si disegni lo schema di uno scrambler additivo avente polinomio caratteristico $1+x^2 +x^3 =x^5$. Quale 
        sarà il massimo periodo della sequenza PRBS generata, avendo tutti ``0'' in ingresso?

        \Problem{25/09/2013}
        \begin{enumerate}
            \item Si disegni la figura di uno scrambler additivo avente polinomio caratteristico $P(x)= 1+x +x^2 +x^4$. 
                Si indichi la sequenza binaria in ingresso con $\{I_k\}$, quella in uscita con $\{U_k\}$ e quella 
                pseudo\nbdash casuale generata con $\{R_k\}$.
            \item Si inizializzi lo scrambler con  ``1'' negli elementi di ritardo $D_1$ e $D_2$ e con ``0'' negli elementi 
                $D_3$ e $D_4$; lo si alimenti con una sequenza in ingresso di soli ``1''. Ricavare la sequenza $\{U_k\}$ 
                all'uscita, evidenziando la sua periodicità: qual è il periodo di tale sequenza?
            \item Verificare che il polinomio $P(x)$ sia irriducibile.
        \end{enumerate}

        \Problem{23/09/2016}
        \begin{enumerate}
            \item Si disegni la figura di uno scrambler additivo avente polinomio caratteristico $P(x)= 1+x +x^3 +x^4$. 
                Si indichi la sequenza binaria in ingresso con $\{I_k\}$, quella in uscita con $\{U_k\}$ e quella 
                pseudo\nbdash casuale generata con $\{R_k\}$.
            \item Si inizializzi lo scrambler con  ``1'' negli elementi di ritardo $D_1$ e con ``0'' negli elementi 
                $D_2$, $D_3$ e $D_4$; lo si alimenti con una sequenza in ingresso di soli ``1''. Ricavare la 
                sequenza $\{U_k\}$ all'uscita, evidenziando la sua periodicità: qual è il periodo di tale sequenza?
            \item Verificare che il polinomio $P(x)$ sia irriducibile.
        \end{enumerate}

    \bigskip
    \section{Scrambler autosincronizzante}
        \Problem{18/07/2014}
        \begin{enumerate}
            \item Si disegni la figura di uno scrambler autosincronizzante avente polinomio caratteristico 
                $P(x)= 1 +x^2 +x^4$. Si indichi la sequenza binaria in ingresso con $\{I_k\}$ e quella in 
                uscita con $\{U_k\}$.
            \item Si inizializzi lo scrambler con  ``1'' negli elementi di ritardo $D_1$ e con ``0'' negli elementi 
                $D_2$, $D_3$ e $D_4$; lo si alimenti con una sequenza in ingresso di soli ``1''. Ricavare la 
                sequenza $\{U_k\}$ all'uscita, evidenziando la sua periodicità: qual è il periodo di tale sequenza?
            \item Verificare che il polinomio $P(x)$ sia irriducibile.
        \end{enumerate}

        \Problem{03/07/2015}
        \begin{enumerate}
            \item Si disegni la figura di uno scrambler autosincronizzante avente polinomio caratteristico 
                $P(x)= 1 +x^4$. Si indichi la sequenza binaria in ingresso con $\{I_k\}$ e quella in 
                uscita con $\{U_k\}$.
            \item Si inizializzi lo scrambler con  ``1'' negli elementi di ritardo $D_i$; lo si alimenti con una 
                sequenza in ingresso di soli ``1''. Ricavare la sequenza $\{U_k\}$ all'uscita, evidenziando la sua 
                periodicità: qual è il periodo di tale sequenza?
            \item Verificare che il polinomio $P(x)$ sia irriducibile.
        \end{enumerate}

        \Problem{30/08/2018}
        \begin{enumerate}
            \item Si disegni la figura di uno scrambler autosincronizzante avente polinomio caratteristico 
                $P(x)= 1 +x +x^3 +x^4$, alimentato da una sequenza di soli ``0'' in ingresso e usato come generatore PRBS. 
                Si indichi la sequenza binaria in ingresso con $\{I_k\}=\{0\}$ e quella in uscita con $\{R_k\}$.
            \item Si inizializzi lo scrambler con $D_i(i=1,\,2,\,3,\,4)$ con $\{1,\,0,\,0,\,0\}$ al passo $k=0$. 
                Ricavare la sequenza $\{R_k\}$ all'uscita, evidenziando la sua periodicità: qual è il periodo di 
                tale sequenza?
            \item Verificare che il polinomio $P(x)$ sia irriducibile. Se lo fosse, quali sarebbero i valori 
                possibili per il periodo?
        \end{enumerate}

        \Problem{11/02/2019}
        \begin{enumerate}
            \item Si disegni la figura di uno scrambler autosincronizzante avente polinomio caratteristico 
                $P(x)= 1 +x +x^2 +x^4$, alimentato da una sequenza di soli ``0'' in ingresso e usato come generatore PRBS. 
                Si indichi la sequenza binaria in ingresso con $\{I_k\}=\{0\}$ e quella in uscita con $\{R_k\}$.
            \item Si inizializzi lo scrambler con $D_i(i=1,\,2,\,3,\,4)$ con $\{1,\,1,\,0,\,0\}$ al passo $k=0$. 
                Ricavare la sequenza $\{R_k\}$ all'uscita, evidenziando la sua periodicità: qual è il periodo di 
                tale sequenza?
            \item Verificare che il polinomio $P(x)$ sia irriducibile. Se lo fosse, quali sarebbero i valori 
                possibili per il periodo?
        \end{enumerate}

        \Problem{02/07/2019}
        \begin{enumerate}
            \item Si disegni lo schema di un generatore PRBS basato su registro a scorrimento LFSR, realizzato con uno 
                scrambler autosincronizzante avente polinomio caratteristico $P(x)= 1 +x^2 +x^4$, alimentato da una 
                sequenza di soli ``0'' in ingresso. Si indichi la sequenza binaria in ingresso con $\{I_k\}=\{0\}$ e 
                quella in uscita con $\{R_k\}$.
            \item Si inizializzi lo scrambler con $D_i(i=1,\,2,\,3,\,4)$ con $\{0,\,1,\,0,\,0\}$ al passo $k=0$. 
                Ricavare la sequenza $\{R_k\}$ all'uscita, evidenziando la sua periodicità: qual è il periodo di 
                tale sequenza?
            \item Verificare che il polinomio $P(x)$ sia irriducibile. Se lo fosse, quali sarebbero i valori 
                possibili per il periodo?
        \end{enumerate}
